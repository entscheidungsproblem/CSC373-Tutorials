\documentclass[pdf]{article}
\usepackage{listings}
\usepackage{xcolor}
\usepackage{amssymb}
\usepackage{hyperref}
%opening
\title{}
\author{}

\begin{document}

\maketitle

\section{Warmup Runtime Questions}
\subsection{Informal}
What is the runtime of:
\begin{enumerate}
	\item Searching for a element in an unordered list?
	\item BFS
	\item Checking if a number is even or odd?
	\item Travelling Salesman?
	\item Searching for element in balanced binary search tree?
\end{enumerate}
\section{Formal}
Formal definition of Big Oh:
\[\exists c, n_0 : \forall n > n_0: f(n) \leq cg(n)\]

\begin{enumerate}
	\item Prove that $f(n)=O(n^2)$, given \[f(n)=O(n+g(n))\] \[g(n) = O(n^2)\]
	\item Prove that $f(n)=O(n^3)$, given \[f(n)=O(n*g(n))\] \[g(n) = O(n^2)\]
	\item Prove that $f(n)=O(\log n)$, given \[f(n)=O(\log (g(n)))\] \[g(n) = \frac{2}{3}n + 20c\] 
	
		
\end{enumerate}

\section{Sorting Algorithms}
Sorting algorithms are given an unsorted list of elements and output a list with the same elements, now sorted by some key. We can assume that the input is a list of unique integers without loss of generality.\\
\href{https://visualgo.net/en/sorting}{Online algorithm visualizers} can be helpful to remind yourself how they work.
\subsection{InsertionSort}
\lstset{language=Python,
	basicstyle=\ttfamily\scriptsize,
	keywordstyle=\color{blue}\ttfamily,
	stringstyle=\color{red}\ttfamily,
	commentstyle=\color{green}\ttfamily,
	breaklines=true,
	numbers=left
}
\begin{lstlisting}
def InsertionSort(lst):
	\\ 1st loop
	for (i in range(1,len(lst)-1)):
		j = i - 1
		\\ 2nd loop
		while (j > 0 and lst[j]<lst[i]:
			j-=1
		lst.insert(j, lst.pop(i))
	return lst
\end{lstlisting}	
How can we analyze the runtime? Try to find the worse case: Reverse sorted. Ex: $8,7,6,5,4,3,2,1$\\
Draw a $n*n$ grid to show the maximum number of operations. For each element in the outer loop, $i$, it is potentially compared to every element before it, $j$. The first $i$ only is compared once so fill in 1 box in the first row of the grid (from the left). Next, the second $i$ can be compared twice, so fill in 2 boxes. Continue to show how the number of operations can be represented by a triangle in the grid (looks like lower triangular matrix). 
\[1+2+3+4+...+n-1+n = \frac{n(n+1)}{2} = O(n^2) \]
\end{document}
